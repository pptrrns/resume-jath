%%%%%%%%%%%% Document
\documentclass[letter]{resume}
\usepackage[letterpaper, left=0.6in,top=0.6in,right=0.6in,bottom=0.6in]{geometry}

\usepackage{color}            % allows usage of color fonts
\definecolor{blue1}{RGB}{31,97,242} % defines color and its name
\usepackage[colorlinks=true, allcolors=blue1, pdfborderstyle={/S/U/W 1}]{hyperref}

\usepackage{vwcol}
\usepackage{natbib}
\usepackage{cfr-lm}           % make old style numbers default
\usepackage{graphicx}
\usepackage{upquote}

\usepackage{setspace}         %allows to change linespacing
\setstretch{1}                %linespacing
\clubpenalty=10000            % avoid clubs
\widowpenalty=10000           % avoid widows

\begin{document}

%%%%%%%%%%%% Title Section
\begin{center}
\name{José Ángel Torrens Hernández}
\contact {+52 55$\cdot$32$\cdot$46$\cdot$78$\cdot$20}
{\href{mailto://jose.torrens@itam.mx}{jose.torrens@itam.mx}}
{\href{https://www.linkedin.com/in/jtorrensh/}{in/jtorrensh}}
{\href{https://github.com/pptrrns}{github/pptrrns}}
{\bf Politólogo, Ciudad de México}

\emph{Intereses}: {Public Policy, Government Affairs, Public Affairs {\em\&} Corporate Affairs.}
\end{center}

%%%%%%%%%%%% Educación 
\section{Estudios}
\begin{content}
{\bf Instituto Tecnológico Autónomo de México} \emph {} \hfill {\bf 2018 -- 2023} \\ 
{ITAM, Licenciatura en \emph{Ciencia Política}} \hfill \emph {Ciudad de México}

\emph{Disponibilidad}: {Tiempo completo a partir de enero de 2024.}

\sectionlineskip
\end{content}

%%%%%%%%%%%% Experiencia
\section{Experiencia}
\begin{content}

\begin{position}{Coca–Cola FEMSA}{Jul 2022 -- actualdiad}{Intern Asuntos Regulatorios y Relaciones Institucionales}{}{}

  \item {\emph{Áreas}}: Asuntos Públicos, Asuntos Regulatorios, Relaciones Institucionales.
  \item Monitoreo de riesgos y tendencias regulatorias para la industria de bebidas embotelladas en 10 países de América Latina.
  \item Elaboración de insumos como \emph{policy brief}, presentaciones, reportes, informes y demás documentos para la estrategia de defensa de categoría de la compañía. Estos insumos incluyen argumentos y narrativas en defensa de la industria para mitigar las tendencias regulatorias que ponen en riesgo la continuidad del negocio. 
  \item Elaboración de reportes de coyuntura, alertas regulatorias e informes de perspectivas políticas, económicas y sociales, a nivel regional e internacional para miemnbros del C-suite.
  \item Gestión del repositorio institucional del área.

\end{position}
\vspace{-.0001 \baselineskip}

\begin{position}{Departamento de Ciencia Política, ITAM}{Ene 2022 -- Dic 2022}{Asistente de Investigación}{Prof. Eric Magar Meurs}{}

  \item {\emph{Áreas}}: Política Comparada, Elecciones.
  \item Análisis de bases de datos utilizando Rstudio y CartoDB.
  \item Creación de mapas y diagramas para el análisis de datos.
  \item Recolección, procesamiento y limpieza de bases de datos.
  \item Gestión de bases de datos utilizando Git y Github.

\end{position}
\vspace{-.0001 \baselineskip}

\sectionlineskip
\end{content}

%%%%%%%%%%%%  Proyectos
\section{Proyectos \textbf{\em\&} Extracurriculares}

\begin{content}
\begin{position}{La Gaceta de Ciencia Política}{Feb 2022 -- actualidad}{\normalfont\href{https://gacetadecpol.com/}{\texttt{gacetadecpol.com}}}{\emph{Consejo Editorial}}{}
  {\emph{Áreas de prácticas}}: Investigación, escritura académica y diseño editorial.
\end{position}

\begin{position}{Modelos de Naciones Unidas}{2013 -- 2017}{\normalfont\href{https://www.linkedin.com/in/jtorrensh/}{\texttt{in/jtorrensh}}}{\emph{Delegado, Presidente, Moderador, Organizador, Staff, etc.}}{}
  \item Participé durante más de 5 años en la planeación, dirección y organización de proyectos de debate para jóvenes universitarios, destacando mi participación como Comité Ejecutivo en más de dos ocasiones del \emph{Modelo de Naciones Unidas del H. Congreso de la Unión}, Congresmun.

\end{position}

\sectionlineskip
\end{content}

%%%%%%%%%%%%  Publciaciones 
%%% \noindent
%%% \renewcommand{\refname}{Publications}   % name of the reference section
%%% \bibliography{biblio.bib}               % bib file
%%% \bibliographystyle{abbrvnat}            % citation style
%%% \nocite{*}
%%% \sectionlineskip

%%%%%%%%%%%%  Cursos
\section{Cursos \textbf{\em\&} Certificaciones}

\begin{content}
\begin{position}{Instituto Nacional de Estadística y Geografía}{2017}{Cursos}{}{}
\item \emph{Mapa Digital de México}, Dirección de Capacitación y Calidad (INEGI, Jan 2017).
\item \emph{Productos de Información Estadística y Geográfica de México}, Dirección de Capacitación y Calidad (INEGI, Jan 2017).
\item \emph{Directorio Estadístico Nacional de Unidades Económicas}, Dirección de Capacitación y Calidad (INEGI, 2017).
\end{position}
\end{content}

%%%%%%%%%%%% Technical Skills
\section{Skills}

\begin{content}
\begin{tabular}{ @{} >{\bf}l @{\hspace{6ex}} l }
  Lenguajes de programación & RStudio (intermedio), \LaTeX\ (intermedio). \\ 
  Herramientas computacionales & MS Office (avanzado), Git \& Github (intermedio), CartoDB (intermedio). \\
  Idiomas & Español (nativo), Inglés (intermedio). \\ 
\end{tabular}

\sectionlineskip
\end{content}

\end{document}