\documentclass[letter]{resume}
\usepackage[letterpaper, 
            left = 1.5cm,
            top = 2cm,
            right = 1.5cm,
            bottom = 1.5cm]{geometry}

\usepackage{color}
\definecolor{teal}{RGB}{16, 96, 89}
\definecolor{azure}{rgb}{0.0, 0.5, 1.0}
\definecolor{red}{RGB}{148, 25, 4}
\usepackage[colorlinks = true,
            allcolors = red,
            pdfborderstyle = {/S/U/W 1}]{hyperref}

\usepackage{vwcol}
\usepackage{natbib}
\usepackage{cfr-lm}
\usepackage{graphicx}
\usepackage{upquote}

\usepackage{setspace}
\setstretch{1}
\clubpenalty = 10000
\widowpenalty = 10000

\begin{document}

%%%%% %%%%% %%%%% Title Section

\begin{center}
\name{José Á. Torrens Hernández}
\normalsize
\contact {+52 55$\cdot$32$\cdot$46$\cdot$78$\cdot$20}
{\href{mailto://jose.torrens@itam.mx}{jose.torrens@itam.mx}}
{\href{https://www.linkedin.com/in/jtorrensh/}{in/jtorrensh}}
{\href{https://github.com/pptrrns}{github/pptrrns}}
\bfseries{Politólogo, Ciudad de México}

% {\emph{Intereses}: Política Pública, Inferencia Causal, Análisis de Datos, Asuntos Gubernamentales \emph{\&} Asuntos Públicos.}
% \vspace{0.05cm}
\end{center}

\spacing{1}

%%%%% %%%%% %%%%% Educación

\section{Estudios}

\begin{content}
% \vspace{0.05cm}

{\bf Instituto Tecnológico Autónomo de México} \hfill{2018 -- 2024}\\ 
{Licenciatura en \emph{Ciencia Política}} \hfill \emph{CGPA: 3.5/4.0}

\sectionlineskip
\end{content}

%%%%% %%%%% %%%%% Experiencia

\section{Experiencia}

\begin{content}
% \vspace{0.05cm}

\begin{position}{Analista de Machine Learning}{Mayo 2024 -- present}{Coca--Cola FEMSA}{}{}
  \item Growth / Run \& Advanced Analytics.
\end{position}

\begin{position}{Intern Asuntos Regulatorios y Relaciones Institucionales}{Julio 2022 -- Mayo 2024}{Coca--Cola FEMSA}{}{}
 \item Análisis y monitoreo de tendencias regulatorias en la industria de bebidas embotelladas en 10 países de América Latina, identificando riesgos y oportunidades clave para la toma de decisiones estratégicas.
 \item Elaboración de \emph{policy briefs}, presentaciones, reportes, informes y demás documentos para la estrategia de defensa de categoría de la compañía. Estos insumos incluyen argumentos y narrativas en defensa de la industria para mitigar las tendencias regulatorias que ponen en riesgo la continuidad del negocio.
 \item Elaboración de reportes de coyuntura, alertas regulatorias e informes de perspectivas políticas, económicas y sociales, a nivel regional e internacional para miembros del \emph{C-suite}.
 \item Investigación de personas e instituciones de interés para la agenda de relacionamiento institucional de la compañía.
\end{position}

\begin{position}{Asistente de Investigación}{2022}{Eric Magar Meurs, Ph.D $\sim$ ITAM}{}{}
 \item Recopilación de información.
 \item Recolección, procesamiento y limpieza de bases de datos.
 \item Gestión de bases de datos utilizando Git y GitHub.
 \item Manejo de bases de datos utilizando RStudio y CartoDB.
 \item Creación de mapas y gráficas para el análisis de datos.
\end{position}
\vspace{-.1 \baselineskip}

\sectionlineskip
\end{content}

%%%%% %%%%% %%%%% Cursos

\section{Cursos \textbf{\em\&} Certificaciones}

\begin{content}
% \vspace{0.05cm}

% \begin{position}{Bloomberg for Education}{feb. 2024}{\normalfont\href{https://portal.bloombergforeducation.com/courses}{\texttt{Bloomberg Market Concepts (BMC)}}}{}{}
% \item Bloomberg Market Concepts (BMC).
% \end{position}

\begin{position}{Evidence in Governance and Politics}{Ago. 2023}{\normalfont\href{https://egap.org/project/learning-days-13-latin-america-regional-hub-workshop/}{\texttt{Latin America Regional Hub (EGAP)}}}{}{}
  \item Participé en los \emph{Días de Aprendizaje del Hub Regional de EGAP en Latinoamérica}, adquiriendo conocimientos sobre el diseño e implementación de Experimentos Aleatorizados Controlados (RCT's) en el contexto de evaluaciones de impacto.
\end{position}

\begin{position}{Instituto Nacional de Estadística y Geografía}{Ene. 2017}{}{}{}
  \item \emph{Mapa Digital de México} (Taller Ejecutivo v.6), {\normalfont\href{https://drive.google.com/file/d/11P-V5IOrVxSAdumvZkqc6cNErlPHsWbA/view?usp=sharing}{\texttt{Dirección de Capacitación y Calidad}}}
  \item \emph{Productos de Información Estadística y Geográfica de México}, {\normalfont\href{https://drive.google.com/file/d/175VzjC148czXa4mLOryoo6QfAnEpCx_z/view?usp=sharing}{\texttt{Dirección de Capacitación y Calidad}}}
\end{position}

\end{content}

%%%%% %%%%% %%%%% Proyectos

\section{Proyectos \textbf{\em\&} Extracurriculares}

\begin{content}
% \vspace{0.05cm}

\begin{position}{La Gaceta de Ciencia Política}{Julio -- Dic. 2022}{\normalfont\href{https://gacetadecpol.wordpress.com}{\texttt{gacetadecpol.com}} $\sim$ \emph{Consejo Editorial}}{}{}
\item Dirigí el Consejo Editorial de \emph{La Gaceta}, una publicación impresa y digital para la difusión de investigaciones en el área de las Ciencias Sociales, particularmente de la Ciencia Política. El objetivo de \emph{La Gaceta} es difundir el trabajo de académicos, estudiantes, activistas y políticos que reflexionan sobre la actualidad de las ciencias sociales.
\end{position}

\begin{position}{Modelos de Naciones Unidas}{2013 -- 2018}{}{}{}
 \item Participé durante más de 5 años en la planeación, dirección y organización de proyectos de debate para jóvenes universitarios, destacando mi participación como Comité Ejecutivo, en más de dos ocasiones, del \emph{Modelo de Naciones Unidas del H. Congreso de la Unión}, Congresmun.
\end{position}

\sectionlineskip
\end{content}

%%%%% %%%%% %%%%% Technical Skills

\section{Technical Skills}

\begin{content}
% \vspace{0.05cm}

\begin{tabular}{ @{} >{\bf}l @{\hspace{6ex}} l }
  Lenguajes de programación & R $\&$ RStudio (Avanzado), Python (Básico), \LaTeX\ (Intermedio). \\ 
  Herramientas digitales & MS Office (Avanzado), Git \& Github (Intermedio), CartoDB (Básico -- Intermedio). \\
  Idiomas & Español (Nativo), Inglés (Avanzado, C1).
\end{tabular}

\sectionlineskip
\end{content}

\end{document}